\documentclass[12pt]{report}
\usepackage{amsmath}
\usepackage{amssymb}
\usepackage{graphicx}
\usepackage{hyperref}
\usepackage{color}
\usepackage{float}
\begin{document}
	\title{A Model of Histone Sliding with UV Dose Dependency}
	\maketitle
	% setting of the model
	We start with a description of DNA and histone loss from the damage region centered at the origin.
	Assuming radial symmetry, our model will described a one-dimensional loss of histone and DNA post UVC. We set $R_0$ to be the radius of the damage region centered at the UV light's focal point, for which damages to the DNA are considered to spread from the origin to $R_0$. We set $R$ to be the radius of the ROI in which chromatin expansion has reached saturation 15 minutes post UVC. Both DNA and histone loss in the ROI will be measured at saturation. 
	
	% 
	We simplify our calculation by considering a single chromatin strand of length $l$, initially compacted in the damage region with one end at the origin and the other at $R_0$. Initially there are $n_0$ histones embedded in $l$. With histones assumed to be uniformly distributed, we have $(R/R_0) n_0$ histones initially in the ROI. 
	
	The expansion of the damaged region is measured according to the position of the most exterior damage point, $s$. 
	
	Chromatin de-compaction  in the damage region results from the recruitment and crowding of repair protein post UVC. As a result of crowding non-damaged DNA and histones are pushed out of the ROI in equal proportions. As chromatin unpacks and recruitment of repair proteins continues, damaged DNA is exposed for repair by sliding histones along the chromatin in a general direction of $R$, away from high concentration of damages. 
	
	The additional loss of histones from the ROI is therefore attributed to histone sliding.  
	 
	By sliding histone over the damage point $s$, the chromatin to the left of $s$ unpacks. At this point, the length of the damaged chromatin up to the right-most damage point does not change, and the point $s$ is at $R$, however the number of histones embedded in $l$ is $n<n_0$. 
	
	Both histone and DNA loss increase with UV dose, however the description of histone and DNA loss is considered to remain similar. Let the ratio $n/n_0 =g$, and the expansion factor $R/R_0=\beta$, DNA and histone loss fractions can be represented as
	\begin{eqnarray}
	d(u) &=& \frac{\beta(u)-1}{\beta(u)}\\
	h(u) &=& d(u) +\frac{n_0-n(u)}{(R(u)/R_0) n_0}=d(u)+\frac{1-g(u)}{\beta(u)}
	\end{eqnarray}
 with $0\leq n(u)\leq n_0$, and $u\geq0$. 
 
The contribution of sliding to the total histone loss is given by
\begin{equation}
h(u)-d(u)=\frac{1-g(u)}{\beta(u)}
\end{equation}

\end{document}