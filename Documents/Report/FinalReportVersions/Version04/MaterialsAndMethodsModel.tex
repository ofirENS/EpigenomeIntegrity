\documentclass[12pt]{article}
\usepackage{amsmath}
\usepackage{amssymb}
%\usepackage{hyperref}
\usepackage{epsfig,graphicx,amsmath}
\usepackage{amsfonts}
\usepackage{enumerate}
\usepackage{amsfonts}
\usepackage{amssymb}
\usepackage{amsthm}


\newcommand{\mb}[1]{\mbox{\boldmath$#1$}}
\newcommand{\p}{\partial}
\newcommand{\ds}{\displaystyle}
\newcommand{\beq}{\begin{eqnarray}}
\newcommand{\beqq}{\begin{eqnarray*}}
\newcommand{\eeq}{\end{eqnarray}}
\newcommand{\eeqq}{\end{eqnarray*}}
\newcommand{\eps}{\varepsilon}
\newcommand{\erf}{\mbox{erf}}
\newcommand{\erfi}{\mbox{erfi}}
\newcommand{\Ei}{\mbox{Ei}}
\newcommand{\x}{\mbox{\boldmath$x$}}
\newcommand{\Aa}{\mbox{\boldmath$A$}}
\newcommand{\rr}{\mbox{\boldmath$r$}}
\newcommand{\As}{\mbox{\boldmath$a$}}
\newcommand{\y}{\mbox{\boldmath$y$}}
\newcommand{\z}{\mbox{\boldmath$z$}}
\newcommand{\J}{\mbox{\boldmath$J$}}
\newcommand{\ET}{\mbox{\boldmath$\eta$}}
\newcommand{\n}{\mbox{\boldmath$n$}}
\newcommand{\X}{\mbox{\boldmath$X$}}
\newcommand{\Y}{\mbox{\boldmath$Y$}}
\newcommand{\Yy}{\mbox{\boldmath$y$}}
\newcommand{\Z}{\mbox{\boldmath$Z$}}
\newcommand{\w}{\mbox{\boldmath$w$}}
\newcommand{\vv}{\mbox{\boldmath$v$}}
\newcommand{\bb}{\mbox{\boldmath$b$}}
\newcommand{\Bb}{\mbox{\boldmath$b$}}
\newcommand{\B}{\mbox{\boldmath$B$}}
\newcommand{\ALPHA}{\mbox{\boldmath$\alpha$}}
\newcommand{\aaa}{\mbox{\boldmath$a$}}
\newcommand{\C}{\mbox{\boldmath$C$}}
\newcommand{\SSigma}{\mbox{\boldmath$\Sigma$}}
\newcommand{\mmu}{\mbox{\boldmath$\mu$}}
\newcommand{\IIm}{\mbox{\boldmath$I_m$}}
\newcommand{\mean}[1]{\langle #1\rangle}
\newcommand{\diffunit}{$\mu$m$^2$.s$^{-1}$}
\newcommand{\Li}{\mbox{Li}}
\newcommand{\thet}{\mbox{\boldmath$\theta$}}
\newcommand{\intR}{\int\limits_{\mathbb{R}}}
\newcommand{\intRm}{\int\limits_{\mathbb{R}^m}}
\newcommand\norm[1]{\left\lVert#1\right\rVert}
%\definecolor{red}{rgb}{1,0,0}

\usepackage{color}
\usepackage{float}
\begin{document}	
\title{ Material and methods}
\maketitle

\section{Result section : Modeling Nucleosome reorganization after damages}

To estimate the nucleosome reorganization following DNA damages, we have constructed a model where redistribution can be due either to chromatin de-compaction or nucleosome sliding along the chromatin or both of them.  Because the histone signal loss is inaccessible experimentally, we have used the model to assess the relative contribution of these two processes to histone eviction from the region of interest (ROI).

The model (presented in Material and methods) follows the DNA $D(u)$ and nucleosome $H(u)$ fraction of signal loss from ROI and for a given UV does $u$, which are calibrated from the measured  H3.3 signal loss and DNA signal loss respectively (Fig. 3A-D). 


\section{Material and methods: Modeling  histones redistribution following UV damages}
We present here a model for nucleosomes and chromatin re-organization following UV damages. We model the steady-state of signal loss as a function of the of the UV dose, and do not specifically model the mechanism of signal loss in time. 

\subsection{Dynamics of histones following UV damages in the region of interest}
Following the experimental protocol, a circular initial damage region (IDR) induced by the laser beam is centered around the focal point (origin of the coordinates) with a fixed area of $A_0$, for $u\in[0, 100]$. Following laser induction, the tagged damage region expands radially outward and reaches it maximal size of $A(u)$ after 15 minutes. The circular expanded region at the end of expansion is defined as the region of Interest (ROI), in which both histone and DNA signals are measured and compared to values before induction to measure loss signal fraction (see also the empirical definition in XXX). 

We assume that the loss of DNA and nucleosomes signal post UVC is due to two mechanisms: the first is chromatin expansion, and the second is histone sliding along the chromatin. In the first mechanism, recruitment and binding of repair factors to damaged DNA causes chromatin de-compactation and expansion of the IDR. Because the majority of damages are inflicted around the laser's focal point the expansion will cause a radial outward pushing force, which result in DNA and histone extrusion from the ROI in equal proportions. In the second mechanism, repair factors expose histone-wrapped damaged DNA by sliding histones along the chromatin to facilitate efficient repair of DNA damages. Sliding of histones out of the IDR in the general direction of the ROI boundary causes expansion of the IDR (seen by the tagged damage or repair proteins) but little or no DNA signal loss. In our model the ROI 15 minutes post UVC and the IDR contain the same amount of DNA. Thus, sliding of histones allows to break the symmetry between histone and DNA signal loss and allow greater histone signal loss. 

\subsection{Fraction of DNA and nucleosome loss }\label{subsection:fractionOfDNAandNucleosomeLoss}
We assume an initial uniform distribution of both DNA and nucleosome. 
In the model, we set the number of histones in the IDR as $N(u)$. The ROI is considered to be a 2D circular region with an area $A(u)$, and the circular IDR of area $A_0$. 
We shall now compute the fraction of DNA loss (resp. histones) $D(u)$ (resp. $H(u)$) in the ROI 15 minutes post UVC. 

By construction, $D(u)$ is given by the ratio of the amount of DNA in the annulus between the ROI and $A_0$ to the total amount of signal in the ROI, while $H(u)$ is the sum of the nucleosomes that have been translocated with the DNA plus the ones that are sliding out, resulting in the following formulas
\begin{eqnarray}
D(u)&=& \frac{(A(u)/A_0) -1}{(A(u)/A_0)} \\
H(u)&=&D(u)+\frac{N(0)-N(u)}{N(0)(A(u)/A_0)}.
\end{eqnarray}
In order to evaluate the functions above, we will now construct models for the functions $A(u)$ and $N(u)$. The accumulation of DNA damages with uv dose is thought of as governing the dynamics of both histone and DNA signal loss. Therefore, we start our construction with a description of the accumulation of damages $T(u)$ in the IDR, and derive the function $N(u)$ and $A(u)$ from it.

\subsection{Deriving the number of damages $T(u)$}
We assume here that the rate of accumulating DNA damages, $T(u)$, with increasing uv dose in the IDR is increasing proportional to the undamaged DNA in the IDR.
\begin{equation}
\frac{dT(u)}{du}=k_T\left(T_{max}-T(u)\right)
\end{equation}
with $k_T$ the rate constant, and $T_{max}$ the maximal number of damages possible in $A_0$. 
Using the initial condition $T(0) = 0$ the solution is
\begin{equation}
T(u) = T_{max}\left(1-\exp(-k_T u)\right) 
\end{equation}
We assume no two damages can occur in the same position on the DNA, hence we can treat the quantity 
\begin{equation*}
T(u)/T_{max}
\end{equation*}
as the fraction of chromatin length in the IDR which is damaged, or as the DNA damage -coverage percentage. 

\subsection{Deriving the function $A(u)$ and $N(u)$}
We now turn to construct a model for the number of histones $N(u)$ left in ROI, as a function of the UV dose. Although the exact mechanism by which histones are lost is not known, we assume here that the number of histones leaving the ROI, $N_0-N(u)$, is proportional to the rate of accumulation of DNA damages $T(u)$ on nucleosomes. The nucleosmes occupy a length of the chromatin proportional to $N(u)$, whereas chromatin damage coverage is proportional to $T(u)$. Therefore, in the first-order approximation, the dynamics of $N(u)$ is given as 

\begin{equation*}
\frac{dN(u)}{du} = -k_N\frac{N(u)}{T_{max}}\frac{dT(u)}{du}
\end{equation*}
with $k_N$ a constant describing the rate of histone depletion from the ROI due to sliding. Using the initial condition $N(0) = N_0$, the solution is given by
\begin{equation}\label{eq:NumHistones}
N(u) = N_0\exp\left(-k_N\frac{T(u)}{T_{max}})\right).
\end{equation}

Next, we model the dynamics of the ROI area $A(u)$ with increasing uv dose.  For this end, we consider the ROI expansion to be affected by two additive mechanisms: one is chromatin de-compaction, and the other is histone sliding. In the first mechanism, an increase in the number of damages leads to greater chromatin expansion, due to the opening of chromatin cross-links and increase in density of repair factors in the damaged region which create pushing radially outward.  In the second mechanism, damaged DNA is spatially displaced radially outward due to sliding of histones over each damaged position of the DNA. 

\begin{equation}\label{dralpha}
\frac{dA(u)}{du}=-k_A\frac{dN(u)}{du}+k_B\frac{dT(u)}{du}
\end{equation}
where $k_A$ is a constant. Using the initial condition $A(0)=A_0$, we find 
\begin{equation}\label{eq:RoiArea}
A(u) = A_0 +k_AN_0\left(1-\frac{N(u)}{N_0})\right) +k_BT(u)
\end{equation}
with $A(0)=A_0$ represents an area equal to $A_0$  with no UV induction. 

We can now substitute equations \ref{eq:NumHistones} and \ref{eq:RoiArea} into the formula for histone and DNA signal loss to get the expressions 
\begin{equation}
\label{eq:DnaLoss}
D(u) = \frac{k_AN_0\left(1-\exp\left(-k_NT(u)/T_{max}\right)\right) +k_BT(u)}{A_0+k_AN_0\left(1-\exp\left(-k_NT(u)/T_{max}\right)\right) +k_BT(u)}
\end{equation}
and 
\begin{equation}\label{eq:histoneLoss}
H(u) = 1- \frac{\exp\left(-k_NT(u)/T_{max}\right)}{1+\frac{k_AN_0}{A_0}\left(1-\exp\left(-k_NT(u)/T_{max}\right)\right) +\frac{k_BT_{max}}{A_0}T(u)}
\end{equation}


\subsection{Parameter fit for $H(u)$ and DNA $D(u)$ loss}\label{subsection:parameterFit}
We now use eqs. \ref{eq:histoneLoss} to fit the experimental data describing the fraction of histone loss from the ROI 15 minutes post UVC. Because $D(u)$ and $H(u)$ share similar parameters, only the H3.3 data will be used to fit the function $H(u)$. 
By classical fitting procedure we find a $R^2= 0.94$ using the following parameters
\begin{eqnarray*}
k_T &=&  0.021\\
k_N &=&  0.39\\
\frac{k_AN_0}{A_0}&=& 0.44\\
\frac{k_BT_{max}}{A_0}&=& 0.23
\end{eqnarray*}

Plugging these parameters into equation \ref{eq:DnaLoss} and calculating the deviation we find a value of $R^2=0.87$
\begin{figure}[H]
\centering
\includegraphics[width=0.6\linewidth, height=0.3\textheight]{histoneAndDnaVsUvDoseModelFit}
\caption{\textbf{Histone loss (red): experimental data (circle) versus analytical curve (continuous)}. The fit is obtained from  eq. \ref{eq:histoneLoss}.  The parameters found in fitting of the H3.3 loss signal are plugged into equation \ref{eq:DnaLoss} for DNA signal loss (green dashed curve) and plotted against experimental points (green triangles).}
\label{fig:histoneAndDnaVsUvDoseModelFit}
\end{figure}


\subsection{Relative contribution of opening and sliding to DNA and histone loss}
Using the analytical expression for histone and DNA loss and the fitted parameters, we now   calculate the relative contribution of chromatin opening and histone sliding to the total loss of DNA and histones.
We start with the contribution of histone sliding and chromatin opening to the expansion of the ROI 15 minutes post UVC. The ROI area can be presented in the following way
\begin{equation*}
A(u) = A(u)_{opening}+A(u)_{sliding}
\end{equation*}

The effect of de-compaction on the expansion of the damage region and loss of signal is given by the second term on the right hand-side of equation \ref{dralpha}. We can therefore zero out the sliding term and solve the equation for the ROI area to get the opening contribution 
\begin{equation*}
\frac{dA(u)_{opening}}{du}=k_B\frac{dT(u)}{du}
\end{equation*}
with the initial condition $A(0)_{opening}=A_0$, the solution is 
\begin{equation}
A(u)_{opening}= A(0)+k_BT_{max}(1-\exp(-k_Tu))
\end{equation}
The fraction attributed to opening out of the total DNA loss due is given by 
\begin{equation}
\frac{D(u)_{opening}}{D(u)}=\frac{\left(A(u)_{opening}/A_0\right)-1}{\left(A(u)/A_0\right) -1}
\end{equation}
and the complementary function 
\begin{equation}
\frac{D(u)_{sliding}}{D(u)}=1-\frac{D(u)_{opening}}{D(u)}
\end{equation}
indicates the relative contribution of sliding to the total DNA loss. 

Similarly, the relative contribution of opening to the total histone loss is given by 
\begin{equation}
\frac{H_{opening}}{H(u)} =\frac{\left(A(0)_{opening}/A_0\right)-1}{(A(u)/A_0)-(N(u)/N_0)}
\end{equation}
and for sliding
\begin{equation}
\frac{H_{sliding}}{H(u)} = 1-\frac{H(u)_{opening}}{H(u)} 
\end{equation}
	
where here we used the fact that $H(u)_{opening}=D(u)_{opening}$

\begin{figure}[H]
	\includegraphics[width=0.5\linewidth, height=0.3\textheight]{relatiiveContributionToDNALoss}
	\includegraphics[width=0.5\linewidth, height=0.3\textheight]{relativeContributionToHistoneLoss}
	\caption{\textbf{Relative contribution of chromatin opening and histone sliding to DNA (left) and histone (right) loss}. }
	\label{fig:relatiiveContributionToDNALoss}
\end{figure}
\end{document}

