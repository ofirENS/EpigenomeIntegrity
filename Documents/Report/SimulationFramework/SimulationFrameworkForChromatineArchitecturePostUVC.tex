\documentclass[12pt]{report}
\usepackage{amsmath}
\usepackage{amssymb}
\usepackage{graphicx}
\usepackage{hyperref}
\usepackage{color}
\usepackage{float}
\begin{document}	
	\title{Simulation framework for chromatin architecture post UV-C irradiation}
	\section{The biological process}
	
	\section{The simulation framework}
	
	\subsection{The simulation domain}
	All simulation are performed in a two-dimensional open domain. The maximal z-projection values of DNA and nucleosome signals measured in experiments [REF].Our simulation framework is constructed to describes the dynamics of the chromatin post UV-C. All dynamics takes place in the local micro-environment of the UV beam. 
	
	\subsection{The polymer model}
	We use a cross-linked Gaussian chain [REF] of $N$ monomers connected by harmonic springs to represent a coarse-grained model of the chromatin. Springs connecting adjacent monomers in the linear chain are assigned a spring constant $2k_BT/b^2$ and a minimal length $L_0$.  Cross links are represented as harmonic springs having spring constant $2k_BT/b^2$ and a minimal length zero. Cross-linking measure is given by the percentage $0\leq \alpha\leq 100$ of non-nearest-neighbor monomers connected out of the $N$ monomers of the chain. For each realization of the polymer, cross-links are added between non nearest-neighbor pairs of monomers chosen uniformly at random. The cross-linked polymer is then simulated up to its relaxation time, set to be the slowest mode of the Gaussian chain [REF], after which the UV beam is shot.
	
	\subsection{UV irradiation}
	At the end of relaxation steps the UV beam's focal point is placed at the polymer's center-of-mass where the beam is shot. A damage region (DR) is set to be a fixed two-dimensional circular region of area $A_0$ and centered at laser's focal point. Damages to DNA are represented by labeling monomers as damaged. For each UV dose $u$, damages caused by UV are uniformly distributed  between the monomers located in $A_0$ at the time of beam shot. With increase of UV dose, we increase the probability of damages linearly as $k_tu$, with $k_t$ in units of $bp/msec$. 
	
	\subsubsection{Affect of UV irradiation on the polymer, the repair stage}	
	To simulate the affect of repair proteins crowding at sites of DNA damage, a circular exclusion region is centered at each damaged monomer. The exclusion region is represented by an elastic spring pushing force of radius $r_p$, originating from each damaged monomer. The elastic force applied on any monomer within the exclusion range is thus oriented outwards. In addition to the exclusion region, all cross-links from and to damaged monomers are removed.
	
	The polymer will evolve into a new steady spatial configuration which represents the chromatin 15 minutes post UV-C. At which point, the region of interest (ROI) is defined as the circle containing 95\% of the damaged monomers and centered at their center-of-mass. The ROI remains a fixed region used to track the number of damaged and undamaged monomers within it. Measurements are done off-line, such that the ROI is always centered at the center-of-mass of the monomers known to be damaged.
	
	\subsection{Post repair stage}
	As damaged monomers are repaired the exclusion region is removed from damaged monomers. Cross-links are reintroduced gradually according to the spatial distance between monomers. The amount of cross-links re-introduces is such that the initial cross-linking percentage $\alpha$ is restored.
	
	
\end{document}