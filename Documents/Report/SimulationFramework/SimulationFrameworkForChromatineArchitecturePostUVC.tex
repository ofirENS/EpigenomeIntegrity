\documentclass[12pt]{report}
\usepackage{amsmath}
\usepackage{amssymb}
\usepackage{graphicx}
\usepackage{hyperref}
\usepackage{color}
\usepackage{float}
\begin{document}	
	\title{Simulation framework for chromatin architecture post UV-C irradiation}
	\section{The biological process}
	
	\section{The simulation framework}
	
	\subsection{The simulation domain}
	A UV beam is shot vertically thorough the nucleus and considered to affect all chromatin in its path similarly. The simulation in therefore placed in a 2-dimensional domain. This is in-line with the experimental signal values, reported as values of a maximal z-projection [REF].
	
	\subsection{The polymer model}
	We use a cross-linked Gaussian chain [REF] of $N$ monomers connected by harmonic springs to represent a coarse-grained model of the chromatin. Springs connecting adjacent monomers in the linear chain are assigned a minimal length $L_0$. Cross-linking are added randomly between pairs of monomers by a harmonic spring with minimal length zero. Cross-linking measure is given by the percentage $\alpha$ of non-nearest-neighbor monomers connected of the $N$ monomers of the chain. In each realization of the simulation, a random set of non-nearest neighbor monomers is chosen for cross-linking, according to the value set for $\alpha$. The cross-linked polymer is simulated up to relaxation time, at which time the UV beam is shot.
	
	\subsection{UV irradiation}
	At the end of relaxation steps, UV beam focal point is set to the polymer's center-of-mass. The damage region (DR) is represented by a 2-dimensional fixed circular region of area $A_0$ centered at laser's focal point. For each UV dose $u$, damages caused by UV are homogeneously distributed in $A_0$ [REF] among the polymer's monomers. Damaged monomers are chosen randomly, such that the average number of damages in $A_0$ increases as $k_tu$, with $k_t$ in units of $bp/msec$. 
	
	\subsubsection{Affect of UV irradiation on the polymer, the repair stage}	
	To simulate crowding of repair proteins around damaged sites, a circular exclusion region is centered at the location of each damaged monomer. Exclusion region is represented by an elastic spring pushing force of radius $r_p$, originating from each damaged monomer. The elastic force applied on any monomer within the exclusion range is thus oriented outwards. In addition to the exclusion region, all cross-links from and to damaged monomers are removed.
	
	The system will evolve into a new steady configuration which represents the chromatin 15 minutes post UV-C. At which point, the region of interest (ROI) is defined as the circle containing 95\% of the damaged monomers and centered at their center-of-mass. The ROI remains a fixed region used to track the number of damaged and undamaged monomers within it. Measurements are done off-line, such that the ROI is always centered at the center-of-mass of the monomers known to be damaged.
	
	\subsection{Post repair stage}
	As damaged monomers are repaired the exclusion region is removed from damaged monomers. Cross-links are reintroduced gradually according to the spatial distance between monomers. The amount of cross-links re-introduces is such that the initial cross-linking percentage $\alpha$ is restored.
	
	
\end{document}