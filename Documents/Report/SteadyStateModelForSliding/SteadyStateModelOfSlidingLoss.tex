\documentclass[12pt]{report}
\usepackage{amsmath}
\usepackage{amssymb}
\usepackage{graphicx}
\usepackage{hyperref}
\usepackage{color}
\usepackage{float}
\begin{document}	
\title{Steady-State Model for Histone Sliding}
\maketitle


Here we describe the loss of histone due to sliding out of the damage region by using a Michaelis-Menten type of modeling.
The total loss of histone will be given as the steady-state of this model, after loss has reached saturation. 

We set $[P]$ to be the free repair protein in the system, $[D]$ the number of damage sites unbound by repair protein, the complex $[PD]$ as the number of damage sites bounded to repair protein (or DDB2), the number of histones lost are considered as the eventual product of the complex $[PD]$, and $U$ is the UVC dose. In the Michaelis-Menten formulation we have 
\begin{equation}
P+D\rightleftharpoons_{k_r}^{k_f}[PD]\rightarrow^{k_{cat}}P+h
\end{equation}
With the rate constant $k_f$ describing the rate of binding, $k_r$ describes the rate of unbinding  from a repaired damage site initially on , $k_{cat}$ the rate at which histone are slided out of the damage region. 

\end{document}